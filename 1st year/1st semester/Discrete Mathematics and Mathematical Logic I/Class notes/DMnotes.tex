\documentclass[12pt, a4paper]{book}
\usepackage[top=2 cm, bottom=2 cm, left=2 cm, right=2 cm]{geometry}
\usepackage[T1]{fontenc}
\usepackage[utf8]{inputenc}
\usepackage{amsmath}
\usepackage{amssymb}
\usepackage{mathtools}
\usepackage{amsthm}
\usepackage[english]{babel}
\usepackage{parskip}
\usepackage{float}
\usepackage{physics}
\usepackage{graphicx}
\usepackage{multicol}
\usepackage{import}
\usepackage{cancel} 
\usepackage{amsthm}
\usepackage{xifthen}
\usepackage{pdfpages}
\usepackage{calc}
\usepackage{svg}
\usepackage{titling}
\usepackage{array}
\usepackage{ulem}
\usepackage{soul}

\newcommand{\incfig}[1]{%
\def\svgscale{1}
\import{./figures/}{#1.pdf_tex}
}

\newsavebox\CBox
\newcommand\hcancel[2][0.5pt]{%
    \ifmmode\sbox\CBox{$#2$}\else\sbox\CBox{#2}\fi%
    \makebox[0pt][l]{\usebox\CBox}%  
    \rule[0.5\ht\CBox-#1/2]{\wd\CBox}{#1}}

\newcommand\vertarrowbox[3][6ex]{%
  \begin{array}[t]{@{}c@{}} #2 \\
  \left\downarrow\vcenter{\hrule height 0.2pt}\right.\kern-\nulldelimiterspace\\
  \makebox[0pt]{\scriptsize#3}
  \end{array}%
  }


  \begin{document}
  \theoremstyle{plain}
  \newtheorem{thm}{Theorem}[chapter]
  \newtheorem{lem}[thm]{Lemma}
  \newtheorem{prop}[thm]{Proposition}
  \newtheorem*{cor}{Corollary}


  \theoremstyle{definition}
  \newtheorem{defn}[thm]{Definition}
  \newtheorem{conj}{Conjecture}[section]
  \newtheorem{exmp}[thm]{Example}

  \theoremstyle{remark}
  \newtheorem*{rem}{Remark}
  \newtheorem*{note}{Note}


  \title{Discrete Mathematics}
  \author{SyG}
  \maketitle


\chapter{Introduction}

\section*{Propositional logic}

Propositional logic (and mathematics, in general) studies propositions: \textbf{declarative sentences} 
(a sentence that declares a fact) that is either true or false, but not both.

\begin{exmp}
    Propositions:
    \begin{enumerate}
        \item Toronto is the capital of Canada (false but a declarative sentence nonetheless)
        \item 1+1=2
        \item 2+2=3 
        \item 3 is a prime number
    \end{enumerate}
    The following are not propositions:

    Not declarative:
    \begin{enumerate}
        \item What time is it?
        \item Read this carefully.
    \end{enumerate}

    Neither true or false:

    \begin{enumerate}
        \item $x+1-2$
        \item $x+y-z$
    \end{enumerate}
\end{exmp}

We use letters to denote propositions: \textbf{p, q, r, }$\ldots$. Now propositions (called compound propositions) are
constructed by combining one or more propositions using logical operators.

\subsection*{Negation (NOT)}
If p is a proposition, its negation is denoted by $\lnot p$. 

"It's not the case that p". "The negation of p".

\begin{exmp}
    p: "My PC runs Linux"

    $\lnot p$: "It's not the case that my PC runs Linux" $\Rightarrow$ "My PC doesn't run Linux"
\end{exmp}

\begin{exmp}
    p: $1+1=2$
    $\lnot$ p: $1+1 \neq 2$

    $\Rightarrow \ \lnot p$ is true iff (if and only if) p is false.
\end{exmp}

\subsection*{Conjunction (AND)}

Let p, q be two propositions $\Rightarrow$ $p \wedge q$ "p and q".

$p \wedge q$ is true iff p and q are true.


\begin{rem}
    Sometimes the word "but" is used instead of "and". For example: 2 is even but 3 is odd.
\end{rem}

\subsection*{Disjunction (OR)}

Let p, q be two propositions $\Rightarrow$ $p \vee q$ "p or q".

$p \vee q$ is true iff p is true, q is true or both are.

This correspondos to the \textbf{inclusive or} in English.

\begin{rem}
    The \textbf{exclusive or}, it is not possible to have both propositions. For example: soup or salad comes as an entrée, it most certainly
    means that the customer cannot have both soup or salad.
\end{rem}

\subsection*{Conditional statement / Implication}

Let p, q be two propositions $\Rightarrow$ $p \rightarrow q $ "if p, then q".

Because of its essential role in mathematical reasoning, a variety of terminology is used to express $p \rightarrow q$:

\begin{itemize}
    \item if p and then q
    \item if p, q $\rightarrow$ p implies q
    \item q if p $\rightarrow$ p only if q
    \item q when p
    \item if p, q
\end{itemize}

$p \rightarrow q$ is false when p (the hypothesis / antecedent) is true and q (the consequence / conclusion) is false; otherwise, it is true.

Useful way to understand its truth value: A pledge many politicians make when running for office: "If I'm elected, I will lower taxes".
It is only when the politician is elected that does not lower taxes that it can be said he has broken his pledge.

\begin{rem}
    Note that this definition is more general tthan the meaning attached to such statements in English: there needs to be no relationship between p and q: "If the Moon is made of cheese, then 2+3=4".
\end{rem}

\subsection*{Biconditional statement / Bi-implication}

$q \leftrightarrow q$. It's equivalent to $(p \rightarrow q) \wedge (q \rightarrow p)$. 

It's True iff the truth values of $p$ and $q$ are the same.

The main advantage of logical language over natural ones is that it removes ambiguity.

\section*{Predicate logic or first-order logic}

Statements such as: $x>3$ or $x+1=y$ are often found in mathematical assertions. They are neither true or false (when the value of the variables are not specified) and have fall outside of the scope of propositional logic.

In $"x>3"$ there are two parts:
\begin{itemize}
    \item the variable x.
    \item the predicate "is greater than 3".
\end{itemize}

By denoting the predicate with P, the statement can be represented as $P(x)$.

\begin{exmp}
    \begin{itemize}
        \item If $P(x)$ is $"x>3"$ then P(4) is True, P(2) is False, P(y) is undefined.
        \item $Q(x,y)$ is $"x=y+2"$: $Q(3,1)$ is True.
    \end{itemize}
\end{exmp}

\subsection*{Quantifiers}

In addition to assigning values to variables, there is another way to get truth values from predicate statements. For that, it is necessary to assume a domain or universe.

\subsubsection*{Universal quantification}

$$P(x) \ \rightarrow \ \forall x, \ P(x) \ : \ \text{ P(x) for all values of x in the domain}.$$

$$\forall x, \ P(x) \text{ is true iff P(a) is true for all elements a in the universe}$$

\begin{exmp}
    \begin{enumerate}
        \item $\forall x, \ x+1 > x$ is true in $\mathbb{R}$ and $\mathbb{N}$.
        \item $\forall x, \ x \geq 0$ is true in $\mathbb{N}$ but not in $\mathbb{R}$. Alternative, mathematical notation: 
        \begin{itemize}
            \item $\forall x \in \mathbb{N}, \ x \geq 0$
            \item $\forall x \in \mathbb{R}, \ x \geq 0$
        \end{itemize}
        \item $U= \{0, \ 1, \ 2, \ 3 \}, \ \forall x \in U, \ x^2<10$ is true.
        \item $\forall x, \ \forall y, \ \ x+y>x$ is true in $\mathbb{N}$, false in $\mathbb{R}$.
        \item $\forall x, \ \forall y, \ \ x \cdot y >x+y$ is false in $\mathbb{N}$ and in $\mathbb{R}$
    \end{enumerate}
\end{exmp}

\subsubsection*{Existencial quantification}

$$P(x) \ \rightarrow \ \exists x, \ P(x) \ : \ \text{"There exists at least one element x in the domain such that P(x)"}.$$
$$\exists x, \ P(x) \text{ is true iff P(a) is true for some a in the universe}$$

\begin{exmp}
    $\exists x, \ x<0$ is true in $\mathbb{R}$ but not in $\mathbb{N}$.
\end{exmp}

\subsubsection*{Both quantifiers}
\begin{exmp}
    \begin{enumerate}
        \item $\forall x, \ \exists y: \ x+y=0$ is True in $\mathbb{R}$ and False in $\mathbb{N}$.
        \item $\exists y, \ \forall x: \ x+y=0$ is False in $\mathbb{R}$ and True in $\mathbb{N}$.
    \end{enumerate}
\end{exmp}

\begin{rem}
    The order of the quantifiers matters.
\end{rem}

\begin{rem}
    What if the universe is empty (there are no elements)?
    $$\forall x, \ P(x) \text{ is trivially true}$$
    $$\exists x, \ P(x) \text{ is false}$$
\end{rem}

\subsection*{Proof techniques}
\subsubsection*{Direct proofs}
$p \rightarrow q$ can be proved by showing that if p is True then q must be also True.

\begin{exmp}
    \begin{proof} [If n is even then $n^2$ is even. ]
        Assumming n is even, therefore $n=2k, \ k \in \mathbb{N}.$ 
        
        We then have $n^2=4k^2, \ k \in \mathbb{N}$. We can define $i=k^2.$
    
        $\Rightarrow \ n^2=2 \cdot (2i). \Rightarrow$ We define $2i=m$ 
        
        $\therefore n^2=2 \cdot m$ is even.
    \end{proof}
\end{exmp}

\subsubsection*{Proof by contraposition}
To prove $p \rightarrow q$ is equivalent to showing its contrapositive: $\lnot q \rightarrow \lnot p$.

False iff $\lnot q$ is True, $\lnot p$ is False.

$\Leftrightarrow$ q is False, p is True.

$\Leftrightarrow$ p $\rightarrow$ q is False.

\begin{exmp}
    \begin{proof}[If $3n+2$ is odd, then n is odd.]


    We define p="If $3n+2$ is odd" and q="n is odd."

    $\Rightarrow \ \lnot q$="n is even".
    *** terminar

    \end{proof}

\end{exmp}

\subsubsection*{Vacuous and trivial proofs}
$p \rightarrow q$ is True if the hypothesis is False.

$p \rightarrow q$ is True if so is q.

\begin{exmp}
    If $n^2$ is odd then n is odd (contrapositive: start with a draft proof)

    *** terminar
\end{exmp}

\subsubsection*{Proofs by contradiction}

To show that q is True it is enough to show that $\lnot p$ leads to a contradiction.

\begin{exmp}

    \begin{proof}[Out of 22 weekdays, at least 4 must fall on the same day of the week.]    

    Let's first assume that the result is not True i.e. False. 

    If each day is repeated 3 times, we have 21 weekdays in the end. But as we have 22 weekdays, at least one of them is repeated 4 times. Therefore the hypothesis by contradiction is True.

    \end{proof}
\end{exmp}

\subsubsection*{Counterexample}
To show that $\forall x, \ P(x)$ is False, we need to only find a value such that P(a) is False.

\begin{exmp}
    \begin{proof}[Every positive integer is the sum of three squares]
    $$6=1^2+2^2+1^2 \text{ but } 7= 2^2+1^2+1^2+1^2 \text{ (it's four squares)} $$
    \end{proof}

\end{exmp}

We have the following Universe: Kingdom={99 cities}.
Roads join 2 cities and 3 roads go through every city. 

As every road from and to a city, we need to avoid counting it twice, then:
$$\displaystyle\frac{99 \circ 3}{2}$$, 
which isn't an integer, then it is impossible for it to exist. 



% New Unit ==========================================================================================================================================
% New Unit ==========================================================================================================================================
% New Unit ==========================================================================================================================================
% New Unit ==========================================================================================================================================
% New Unit ==========================================================================================================================================
% New Unit ==========================================================================================================================================
% New Unit ==========================================================================================================================================
% New Unit ==========================================================================================================================================
% New Unit ==========================================================================================================================================




\chapter{2: Numbers, induction and recursion}

\section{Principle of Mathematical Induction}
The successor of $n\in\mathbb{N}$ is $s(n)=n+1$.

The set $\mathbb{N}$ of natural numbers is generated by $0$ and $s$.

\[
    \begin{aligned}
        \mathbb{N} = &\{0,1,2,3,\ldots\} \\
        &\{0,s(0),s(s(0)),s(s(s(0)),\ldots)\}
    \end{aligned}
\]

Similarly, starting with $m\in\mathbb{N}$ and applying $s$, the set $\mathbb{N}_m$ of all natural numbers, from $m$ upward, is generated:
\[
    \begin{aligned}
        \mathbb{N} = &\{m,m+1,m+2,\ldots\} \\
        &\{m,s(m),s(s(m)),\ldots)\}
    \end{aligned}
\]

\begin{defn}
    \textbf{Principle of mathematical induction}

    To prove that $P(n)$ is true for all numbers, we complete two steps:
    \begin{itemize}
        \item Basis step: we verify that $P(0)$ is true.
        \item Inductive step: assuming that $P(k)$ is true (inductive hypothesis), we use it to show that $P(k+1)$ is true. \newline
    \end{itemize}
\end{defn}

\begin{exmp} \ 
    \begin{itemize}
        \item $P(n): \forall n \geq 0: 2+4+6+\ldots +2n = n(n+1)$.
        \item $Q(n): \forall n \geq 0: 1+2+2^2+\ldots + 2^n = 2^{n+1}-1$.
    \end{itemize}
\end{exmp}

Sometimes, the property $P(n)$ holds for all numbers, greater than or equal to $w$. In this case, we need to modify the basis step:

\begin{itemize}
    \item Basis step: we verify that $P(w)$ is true.
\end{itemize} 

\begin{itemize}
    \item $\forall n \geq 1: 1+3+5+\ldots +(2n-1) = \sum_{i = 1}^{n}(2i-1) = n^2$
    \item Exercise: $2^n < n!, \forall n \geq 4$.
\end{itemize}

\begin{defn}
    \textbf{Strong induction}

    To prove that $P(n)$ is true for all $n\in\mathbb{N}$, we complete two steps:
    \begin{itemize}
        \item Basis step: we verify that $P(0)$ is true.
        \item Inductive step: assuming that $P(0),P(1),\ldots$ and $P(k)$ are true, we show that $P(k+1)$ is true.
    \end{itemize}
\end{defn}

\begin{itemize}
    \item $\forall n>1$, $n$ can be decomposed as the product of prime factors.
\end{itemize}

Sometimes it is also useful to modify the basis step in the strong induction:
\begin{itemize}
    \item Basis step: we verify that $P(w)$ is true.
\end{itemize}

More exercises:
\begin{itemize}
    \item $\displaystyle \forall n \geq 1: 1+a+a^2+\ldots + a^n = \frac{a^{n+1}-1}{a-1}$.
    \item Every price $\geq 24$ can be paid using notes of $5$ and $6$.

    $\forall n \geq 24, \exists p,q\geq 0: n = 5p + 6q$
\end{itemize}


\section{Recursive definitions}

\begin{defn}
    Let $A$ be a set and $m\in\mathbb{N}$. A \textbf{sequence} (of elements from $A$ indexed by $\mathbb{N}_m$) is a function:
    \[
        f:\mathbb{N}_m \mapsto A
    \]
    We write $a_n$ for $f(n)$ and call $a_n$ a term of the sequence.
\end{defn}

A sequence can be defined via an explicit or a recursive definition.

\begin{defn}
    \textbf{Explicit definition}: give a direct procedure to compute $a_n$.
    \begin{itemize}
        \item $f:\mathbb{N} \mapsto \mathbb{Q}$ with $\displaystyle f(n) = q_n = \frac{1}{2^n}$

        $\longrightarrow \displaystyle 1, \frac{1}{2}, \frac{1}{4}, \frac{1}{8}, \ldots$
        \item $g:\mathbb{N} \mapsto \mathbb{Z}$ with $g(n) = x_n$ defined by cases:
        \[
            x_n = 
            \begin{cases}
                i & \text{if } n \text{ is even }, n=2i, i \geq 0 \\
                j & \text{if } n \text{ is odd }, n=2j-1, j \geq 1
            \end{cases}
        \]
        $0,-1,1,-2,2,-3,3,\ldots$
    \end{itemize}
\end{defn}

\begin{defn}
    \textbf{Recursive definition}: we use two steps:
    \begin{itemize}
        \item Basis step: specify the value of the sequence for the initial segment $a_m,a_{m+1},\ldots,a_l$.
        \item Recursive step: give a rule for finding its value at $l<n$, from its values at smaller natural numbers.
    \end{itemize}
\end{defn}

\begin{exmp}
    Fibonacci Sequence

    $fib: \mathbb{N} \mapsto \mathbb{N}$
    
    RS: $fib(0) = 0,\ fib(1) = 1 $.\newline 
    RS: $fib(n) = fib(n-2) + fib(n-1),\quad n>1$.

    *** pagina 5, el cuadro

    Alternatively, we can compute the values of $fib(n)$ by ``reduction''.

    $
    \begin{aligned}
        fib(4) &= fib(3) + fib(2) \\
        &= fib(2) + fib(1) + fib(1) + fib(0) \\
        &= fib(1) + fib(0) + 1 + 1 + 0 \\
        &= 1 + 0 + 1 + 1+ 0 \\
        &= 3
    \end{aligned}
    $

    Some calculations are repeated, as opposed to tabulation.
\end{exmp}

\subsection{Summation and product}

Summations and products can be defined recursively.

$\sum_{k = 0}^{n} a_k = a_0 + a_1 + \ldots + a_n = F(n)$

\begin{itemize}
    \item B.S: $\sum_{k = 0}^0 a_k = a_0$
    \item R.S: $\sum_{k = 0}^{n+1} a_k = (\sum_{k = 0}^n a_k) + a_{n+1}$
\end{itemize}

Similarly for products. Consider the factorial:

$n! = n\cdot(n-1) \ldots 3\cdot 2\cdot 1 = \prod_{k = 1}^n k$

\begin{itemize}
    \item B.S: $0! = 1$
    \item R.S: $(n+1)! = (n+1)\cdot n!$
\end{itemize}

\begin{exmp}
    \begin{itemize}
        \item $\forall n\geq 3: \sum_{i = 0}^n {fib(i)}^2 = fib(n)\cdot fib(n+1)$
        ??? properties about recursive sequences often require the strong principle of mathematical induction. Though not in this case.
        \item Let $(a_n)$ be defined as:
        \begin{itemize}
            \item B.S: $a_0 = 2,\ a_1 = 1$
            \item R.S: $a_n = a_{n-1} + 2a_{n-2}$ for $n>1$
        \end{itemize}
        $forall n\geq 0: a_n = 2^n + (-1)^n$ ASK TEACHER ***
        Prove $a_n = b_n$. (Strong induction)
    \end{itemize}
\end{exmp}

\subsection{Recursive functions with more than 1 argument}

Recursive definitions can be used whenever a function has one argument ranging over $\mathbb{N}_m$. For example:

$f:\mathbb{N}\mapsto \mathbb{N}$
\begin{itemize}
    \item B.S: $f(0,a) = a$
    \item R.S: $f(n,a) = f(n-1,n\cdot a),\quad n>0$
\end{itemize}
Recursion over the first argument.

To compute the values of such function, use either tabulation or reduction:

$f(3,a) = f(2,3\cdot a) = f(1, 2\cdot 3\cdot a) = f(0, 1\cdot 2\cdot 3\cdot a) = 1\cdot 2\cdot 3\cdot a = 3!\cdot a$

It can be proved, by \underline{mathematical induction}, that $f(n,a) = n!\cdot a$.


\section{Divisibility}

Let $a,b \in\mathbb{Z}$. We say that $b$ \textbf{divides} $a$, or that $a$ is a \textbf{multiple} of $b$, if there is an integer $c$ such that $a = b \cdot c$. The notation $b/a$ denotes that $b$ divides $a$; we write $b \hcancel{/} a$ when $b$ does not divide $a$.

If $b/a$ and $c$ is unique, we write $\displaystyle c = \frac{b}{a}$. We call $c$ the \textbf{quotient}.

\begin{rem}
    \begin{itemize}
        \item $0/0$ but $\displaystyle \frac{0}{0}$ is undefined.
        \item If $b\neq 0$ then $\displaystyle \frac{0}{b} = 0$.
    \end{itemize}
\end{rem}

\subsection{Integer division}

Let $a,b\in\mathbb{Z}, b\neq 0$. If $b \hcancel{/} a$, the (exact) quotient $displaystyle \frac{a}{b}$ does not exist, but an integer division between $a$ (dividend) and $b$ (divisor) can be performed to produce a quotient ($q$) and a remainder ($r$).

\begin{thm}
    \textbf{Division algorithm}

    Given $a\in\mathbb{Z}$ (dividend) and $b\in\mathbb{Z}$ (divisor), $b\neq 0, b>0$, there are \underline{unique} integers $q$ and $r$ such that
    \[
        a = b \cdot q + r \text{ and } 0\leq r < b
    \]
    \begin{proof}
        \underline{Uniqueness}: Assume $q_1, q_2, r_1, r_2 \in\mathbb{Z}$ such that
        \begin{itemize}
            \item $a = b q_1 + r_1,\ 0\leq r_1 < b$
            \item $a = b q_2 + r_2,\ 0\leq r_2 < b$
        \end{itemize}
        Let us show that $q_1 = q_2$ and $r_1 = r_2$. If $q_1 \neq q_2$ then either $q_1 > q_2$ or $q_2 > q_1$. In case $q_1 > q_2$:

        $
        \begin{aligned}
            a &= b q_1 + r_1 = b q_2 + b(q_1 - q_2) + r_1 \\
            &\Rightarrow a - b q_2 = b(q_1 - q_2) + r_1 \\
            &\Rightarrow r_2 = b(q_1 - q_2) + r_1 \geq b + r_1 \geq b
        \end{aligned}
        $
        Contradicting the fact that $r_2 < b$. (Remember that $q_1 \geq q_2, r_1\geq 0$)

        From $q_2 > q_1$ a similar contradiction is reached and hence $q_1 = q_2$. And then:
        \[
            r_1 = a - b q_1 = a - b q_2 = r_2
        \]
    \end{proof}
\end{thm}

\end{document}