\documentclass[12pt, a4paper]{book}
\usepackage[top=2 cm, bottom=2 cm, left=2 cm, right=2 cm]{geometry}
\usepackage[T1]{fontenc}
\usepackage[utf8]{inputenc}
\usepackage{amsmath}
\usepackage{amssymb}
\usepackage{mathtools}
\usepackage{amsthm}
\usepackage[english]{babel}
\usepackage{parskip}
\usepackage{float}
\usepackage{physics}
\usepackage{graphicx}
\usepackage{multicol}
\usepackage{import}
\usepackage{cancel} 
\usepackage{amsthm}
\usepackage{xifthen}
\usepackage{pdfpages}
\usepackage{calc}
\usepackage{svg}
\usepackage{titling}
\usepackage{array}

\newcommand{\incfig}[1]{%
\def\svgscale{1}
\import{./figures/}{#1.pdf_tex}
}
\newcommand\vertarrowbox[3][6ex]{%
  \begin{array}[t]{@{}c@{}} #2 \\
  \left\downarrow\vcenter{\hrule height 0.2pt}\right.\kern-\nulldelimiterspace\\
  \makebox[0pt]{\scriptsize#3}
  \end{array}%
  }


  \begin{document}
  \theoremstyle{plain}
  \newtheorem{thm}{Theorem}[chapter]
  \newtheorem{lem}[thm]{Lemma}
  \newtheorem{prop}[thm]{Proposition}
  \newtheorem*{cor}{Corollary}


  \theoremstyle{definition}
  \newtheorem{defn}[thm]{Definition}
  \newtheorem{conj}{Conjecture}[section]
  \newtheorem{exmp}[thm]{Example}

  \theoremstyle{remark}
  \newtheorem*{rem}{Remark}
  \newtheorem*{note}{Note}


  \title{Discrete Mathematics}
  \author{SyG}
  \maketitle


\chapter{Introduction}

\section*{Propositional logic}

Propositional logic (and mathematics, in general) studies propositions: \textbf{declarative sentences} 
(a sentence that declares a fact) that is either true or false, but not both.

\begin{exmp}
    Propositions:
    \begin{enumerate}
        \item Toronto is the capital of Canada (false but a declarative sentence nonetheless)
        \item 1+1=2
        \item 2+2=3 
        \item 3 is a prime number
    \end{enumerate}
    The following are not propositions:

    Not declarative:
    \begin{enumerate}
        \item What time is it?
        \item Read this carefully.
    \end{enumerate}

    Neither true or false:

    \begin{enumerate}
        \item $x+1-2$
        \item $x+y-z$
    \end{enumerate}
\end{exmp}

We use letters to denote propositions: \textbf{p, q, r, }$\ldots$. Now propositions (called compound propositions) are
constructed by combining one or more propositions using logical operators.

\subsection*{Negation (NOT)}
If p is a proposition, its negation is denoted by $\lnot p$. 

"It's not the case that p". "The negation of p".

\begin{exmp}
    p: "My PC runs Linux"

    $\lnot p$: "It's not the case that my PC runs Linux" $\Rightarrow$ "My PC doesn't run Linux"
\end{exmp}

\begin{exmp}
    p: $1+1=2$
    $\lnot$ p: $1+1 \neq 2$

    $\Rightarrow \ \lnot p$ is true iff (if and only if) p is false.
\end{exmp}

\subsection*{Conjunction (AND)}

Let p, q be two propositions $\Rightarrow$ $p \wedge q$ "p and q".

$p \wedge q$ is true iff p and q are true.


\begin{rem}
    Sometimes the word "but" is used instead of "and". For example: 2 is even but 3 is odd.
\end{rem}

\subsection*{Disjunction (OR)}

Let p, q be two propositions $\Rightarrow$ $p \vee q$ "p or q".

$p \vee q$ is true iff p is true, q is true or both are.

This correspondos to the \textbf{inclusive or} in English.

\begin{rem}
    The \textbf{exclusive or}, it is not possible to have both propositions. For example: soup or salad comes as an entrée, it most certainly
    means that the customer cannot have both soup or salad.
\end{rem}

\subsection*{Conditional statement / Implication}

Let p, q be two propositions $\Rightarrow$ $p \rightarrow q $ "if p, then q".

Because of its essential role in mathematical reasoning, a variety of terminology is used to express $p \rightarrow q$:

\begin{itemize}
    \item if p and then q
    \item if p, q $\rightarrow$ p implies q
    \item q if p $\rightarrow$ p only if q
    \item q when p
    \item if p, q
\end{itemize}

$p \rightarrow q$ is false when p (the hypothesis / antecedent) is true and q (the consequence / conclusion) is false; otherwise, it is true.

Useful way to understand its truth value: A pledge many politicians make when running for office: "If I'm elected, I will lower taxes".
It is only when the politician is elected that (..) not lower taxes that it can be said he has broken his pledge.

\begin{rem}
    Note that this definition is more general tthan the meaning attached to such statements in English: there needs to be no relationship between p and q: "If the Moon is made of cheese, then 2+3=4".
\end{rem}

\subsection*{Biconditional statement / Bi-implication}

$q \leftrightarrow q$. It's equivalent to $(p \rightarrow q) \wedge (q \rightarrow p)$. It's True iff the truth 
values of p and q are the same.

The main advantage of logical language over natural ones is that it removes ???.

\section*{Predicate logic or first-order logic}

Statements such as: $x>3$ or $x+1=y$ are often found in mathematical assertions. They are neither true or false (when the value of the variables are not specified) and have fall outside of the scope of propositional logic.

In $"x>3"$ there are two parts:
\begin{itemize}
    \item the variable x.
    \item the predicate "is greater than 3".
\end{itemize}

By denoting the predicate with P, the statement can be represented as $P(x)$.

\begin{exmp}
    \begin{itemize}
        \item If $P(x)$ is $"x>3"$ then P(4) is True, P(2) is False, P(y) is undefined.
        \item $Q(x,y)$ is $"x=y+2"$: $Q(3,1)$ is True.
    \end{itemize}
\end{exmp}

\subsection*{Quantifiers}

In addition to assigning values to variables, there is another way to get truth values from predicate statements. For that, it is necessary to assume a domain or universe of discover???

\subsubsection*{Universal quantification}

$$P(x) \ \rightarrow \ \forall x, \ P(x) \ : \ \text{ P(x) for all values of x in the domain}.$$

$$\forall x, \ P(x) \text{ is true iff P(a) is true for all elements a in the universe}$$

\begin{exmp}
    \begin{enumerate}
        \item $\forall x, \ x+1 > x$ is true in $\mathbb{R}$ and $\mathbb{N}$.
        \item $\forall x, \ x \geq 0$ is true in $\mathbb{N}$ but not in $\mathbb{R}$. Alternative, mathematical notation: 
        \begin{itemize}
            \item $\forall x \in \mathbb{N}, \ x \geq 0$
            \item $\forall x \in \mathbb{R}, \ x \geq 0$
        \end{itemize}
        \item $U= \{0, \ 1, \ 2, \ 3 \}, \ \forall x, \ x^2<10$ is true.
        \item $\forall x, \ \forall y, \ \ x+y>x$ is true in $\mathbb{N}$, false in $\mathbb{R}$.
        \item $\forall x, \ \forall y, \ \ x \cdot y >x+y$ is false in $\mathbb{N}$ and in $\mathbb{R}$
    \end{enumerate}
\end{exmp}


\end{document}