\documentclass[12pt, a4paper]{book}
\usepackage[top=2 cm, bottom=2 cm, left=2 cm, right=2 cm]{geometry}
\usepackage[T1]{fontenc}
\usepackage[UTF8]{inputenc}
\usepackage{amsmath}
\usepackage{amssymb}
\usepackage{mathtools}
\usepackage{amsthm}
\usepackage[english]{babel}
\usepackage{parskip}
\usepackage{float}
\usepackage{physics}
\usepackage{graphicx}
\usepackage{multicol}
\usepackage{import}
\usepackage{cancel} 
\usepackage{amsthm}
\usepackage{xifthen}
\usepackage{pdfpages}
\usepackage{calc}
\usepackage{svg}
\usepackage{titling}
\usepackage{array}

\newcommand{\incfig}[1]{%
\def\svgscale{1}
\import{./figures/}{#1.pdf_tex}
}
\newcommand\vertarrowbox[3][6ex]{%
  \begin{array}[t]{@{}c@{}} #2 \\
  \left\downarrow\vcenter{\hrule height 0.2pt}\right.\kern-\nulldelimiterspace\\
  \makebox[0pt]{\scriptsize#3}
  \end{array}%
  }

\begin{document}
\theoremstyle{plain}
\newtheorem{thm}{Theorem}[chapter]
\newtheorem{lem}[thm]{Lemma}
\newtheorem{prop}[thm]{Proposition}
\newtheorem*{cor}{Corollary}


\theoremstyle{definition}
\newtheorem{defn}[thm]{Definition}
\newtheorem{conj}{Conjecture}[section]
\newtheorem{exmp}[thm]{Example}

\theoremstyle{remark}
\newtheorem*{rem}{Remark}
\newtheorem*{note}{Note}


\title{Calculus I}
\author{SyG}
\maketitle


\chapter{Introduction}
\begin{defn} \textbf{Mathematics (according to Oxford Eng. Dictionary)} \end{defn}
The abstract science which investigates deductively the conclusions implicit in the 
elementary conception of spacial and numerical relations.
This science can be divided in 6 main topics:
\begin{enumerate}
  \item \textbf{Foundations:} logic, set theory, proof theorems, etc.
  \item \textbf{Algebra:} numbers, arithmetical operations, order theorems. 
  \item \textbf{Analysis:} differentiation, integration, measure, etc.
  \item \textbf{Geometry and topology:} proporties of space, shape, position of figures.
  \item \textbf{Combinatories:} graph theory, partition theory, etc.
  \item \textbf{Applied Mathematics:} computational sciences, probability, the range of applications
  of Mathematics is wide, such as:
  \begin{enumerate}
    \item \textbf{Banking and Finance:} Black-Scholes equation.
    \item \textbf{Aeronautical engineering:} Fluid mechanics, shape design.
    \item \textbf{Chemistry:} Models for proteing folding, thermodynamics.
    \item \textbf{Informatic:} Cryptography, computational algebra, parallel programming, etc. 
  \end{enumerate}
\end{enumerate}


\subsection*{Summary of the program: 1st Semester}
\begin{itemize}
  \item Real numbers
  \item Sequence and series of numbers
  \item Continuity and limit
  \item Differentiation
  \item Integration
\end{itemize}

\chapter{Real numbers and some basic concepts}

\section{Set of points}
We recall here some basical concepts:
\begin{defn}
  A \textbf{set} is a \textbf{collection of distinct objects.}
\end{defn}
\begin{exmp}
  2, 5, 7 are different objects (numbers). They can compose the set $\{2, \ 5, \ 7 \}$, where $\{ \ldots \}$ denotes the
  set composed by the objects $\ldots$.
\end{exmp}

\begin{note}
  If an object $x$ is a member of a set $\theta$, we denote: 
  $$ x \in \theta, \ \text{else we denote } x \notin \theta$$

  \begin{exmp}
    \[
      \theta = \{0, \ 5, \ 7 \}, \ \ \text{if }x=5 \text{ and } y=9:
    \]
    \[
      x \in \theta \text{ and } y \notin \theta
    \]
  \end{exmp}
\end{note}

\begin{defn}
  Considering two sets A and B. If every element of A is a member of B, A is said to be a \textbf{subset} of B, and we denote:
  \boldmath $$ A \subseteq B$$, else we denote 
  $$ A \nsubseteq B $$ 
  Furthermore, if it exists at least one element of B which is not a member of A, A is said to be a \textbf{proper subset} of B, and we denote

  $$ A \subset B $$. \unboldmath
  
\end{defn}

\begin{exmp}
  \[
    A= \{ 1, \ 2, \ 3\} 
  \]
  \[
    B= \{ 0, \ 1, \ 2, \ 3, \ 4 \}
  \]
  \[
    C= \{ 0, \ 1, \ 2 \}
  \]
  \[
    \therefore
    A \subseteq A, \ \ \ A \subset B, \ \ \ A \nsubseteq C
  \]
\end{exmp}

\begin{defn}
  \textbf{Set operators}
  Let A and B be two sets. 
\end{defn}
\boldmath
\subsubsection*{Union: $\cup$}

The \textbf{union} of $A$ and $B$ is the set

\[ 
  A \cup B = \{x/x \in A \vee x \in B \}
\]

\subsection*{Intersection: $\cap$}

The \textbf{intersection} of $A$ and $B$ is the set
\[
  A \cap B = \{x/x \in A \wedge x \in B \}
\]

\subsection*{Complement: $\backslash$}

\[
  A \; \backslash \; B = \{x/x \in A \wedge x \notin B \}
\]
\unboldmath
\begin{exmp}
  $A=\{3,5,7\}, B=\{5,7,10\}$
  \begin{itemize}
    \item $A \cup B = \{3,5,7,10\}$
    \item $A \cap B = \{5,7\}$
    \item $A \backslash B = \{3\}$
    \item $B \backslash A = \{10\}$
  \end{itemize}
\end{exmp}

\subsection*{Geometrical representation}

****

\subsection*{Properties: }

\begin{itemize}
  \item $A \backslash (B \cup C) = (A \backslash B) \cup (A \backslash C)$
  \item $A \backslash (B \cap C) = (A \backslash B) \cap (A \backslash C)$
\end{itemize}

Graphically (Venn diagrams):

***

\begin{note}
  In the case of various n sets denoted by $A_1, \ A_2, \ A_n$, instead of writing:

  \[
    A_1 \cup A_2 \cup \ldots \cup A_n \text{ we write } \cup_{k=1}^{n} A_k
  \]
  or
  \[
    A_1 \cap A_2 \cap \ldots \cap A_n \text{ we write } \cap_{k=1}^{n} A_k
  \]
\end{note}

\begin{exmp}
  \[
    A_1=\{1, \ 2, \ 3 \}, \ \ \ A_2=\{ 5,\ 6, \ 7 \}, \ \ \ A_3= \{1, \ 5, \ 9 \}
  \]
  \[
    \cup_{k=1}^{3} A_k=A_1 \cup A_2 \cup A_3 = \{ 1, \ 2, \ 3, \ 5, \ 7, \ 9 \}
  \]
\end{exmp}

\begin{rem}
We can apply the same notation in case of infinite ($\infty$) numbers of a set $\{ A_1, \ldots, A_{100}, \ldots \}$.
\[
  \displaystyle\cup_{k=1}^{\infty}A_k \ \ \ \ \text{ and } \ \ \ \ \displaystyle\cap_{k=1}^{\infty}A_k
\]
\end{rem}
some examples and the concept of infinity will be defined in the next sections.

\begin{defn}
  The cartesian product of two sets A and B is denoted by \boldmath $A \times B$  and defined as:
  \[
    A \times B = \{ (a, \ b  ) | a \in A \text{ and } b \in B \}
  \]
  \unboldmath
  where (a, b) is called \textbf{ordened pair.} 
\end{defn}

\begin{exmp}
  \[
    A=\{1, \ 2, \ 3 \}, \ \ \ B=\{7, \ 9 \}
  \]
  \[
    A \times B= \{ (1,7), \ (1,9), \ (2,7), \ (2,9), \ (3,7), \ (3,9) \}
  \]
  More properties of sets will be introduced later in this chapter.
\end{exmp}

\subsection*{Some common sets of real points}
Here we only introduce the set of points used in next chapters.

\begin{defn}
\end{defn}
\boldmath
  \begin{itemize}
    \item $\mathbb{R}=\{ \ldots, \ldots, \ -10, \ldots, \ -7, \ldots, \ 0, \ldots, \ 4, \ldots, 1000, \ldots  \}$ is called the set of \textbf{real numbers} which contains \textbf{all positive and negative numbers}.
    \item $\mathbb{N}= \{1, \ 2 , \ 3 , \ldots \}$ is called the set of \textbf{natural numbers} which contains \textbf{all the strictly positive integer numbers.}
    \item $\mathbb{Z}= \{\ldots, -5, \ -4, \ -3, \ -2, \ -1, \ 0, \ 1, \ 2, \ 3, \ldots  \}$ is called the set of \textbf{integer numbers} and contains the \textbf{positive and negative integers.}
    \item $\varnothing = \{\}$ the \textbf{empty set} represents the sets without any elements. 
    \begin{exmp}
      \[
        A=\{1, \ 4 \}, \ \ B=\{3, \ 4 \} \ \ \ \ A \cap B= \varnothing \text{, ie no coincidences between A and B}
      \]
    \end{exmp} 
    \item $\mathbb{Q}= \{x \in \mathbb{R} \ \vert \ x= \frac{m}{n}, \ m \in \mathbb{Z}, \ n \in \mathbb{Z} \text{ and } n \neq 0 \}$ is called the set of \textbf{rational numbers} and contains the \textbf{real numbers that can be written as a quotient of integer numbers}
    \item \textbf{Odd=} $\{ x \in \mathbb{R} \ \vert \ \exists \ k \in \mathbb{N} \text{ st } x=2k+1 \}$ is the set of the \textbf{odd integer numbers}.
    \item \textbf{Even=} $\{ x \in \mathbb{R} \ \vert \exists \ k \ \in \mathbb{N} \text{ st } x=2k \}$ is the set of the \textbf{even integer numbers}.
    \item $\mathbb{C}= \{x +iy \ \vert \ x \in \mathbb{R} \text{ and } y \in \mathbb{R}  \}$ is the set of \textbf{complex numbers}. Note: i denotes the imaginary number that verifies $i^2=-1$.
  \end{itemize}
  \unboldmath

\begin{rem}
  \[
    \varnothing \subseteq \mathbb{N} \subseteq \mathbb{Z} \subseteq \mathbb{Q} \subseteq \mathbb{R} \subseteq \mathbb{C}
  \]
\end{rem}

\begin{defn}
  A \textbf{relation}, noted by $\leq $, is a \textbf{total order} on a set S if it verifies:
  \begin{enumerate}
    \item \textbf{Reflexivity:} $\forall a \in S, \ a \leq a$
    \item \textbf{Antisymmetry:} $a \leq b$ and $ b \leq a \Rightarrow \ a=b$
    \item \textbf{Transitivity:} $a \leq b$ and $b \leq c \Rightarrow a \leq c$
    \item \textbf{Comparability:} $\forall \ a, \b \ \in S$, either $a \leq b$ or $b \leq a$
  \end{enumerate}
\end{defn}

\begin{exmp}
  \begin{itemize}
    \item The relation $\leq$ applied to $\mathbb{R}$ is a total order.
    \item The relation $\subset$ applied to a subset of $\mathbb{R}$ is \textbf{not} a total order. For example, $\{1, \ 2\}$ and $\{2, \ 4\}$ cannot be compared.  
  \end{itemize}
\end{exmp}

\begin{defn}
  A set plus a total order relation is called a \textbf{total ordered set}.
\end{defn}
\end{document}