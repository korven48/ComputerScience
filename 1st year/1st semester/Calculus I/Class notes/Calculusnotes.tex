\documentclass[12pt, a4paper]{book}
\usepackage[top=2 cm, bottom=2 cm, left=2 cm, right=2 cm]{geometry}
\usepackage[T1]{fontenc}
\usepackage[UTF8]{inputenc}
\usepackage{amsmath}
\usepackage{amssymb}
\usepackage{mathtools}
\usepackage{amsthm}
\usepackage[english]{babel}
\usepackage{parskip}
\usepackage{float}
\usepackage{physics}
\usepackage{graphicx}
\usepackage{multicol}
\usepackage{import}
\usepackage{cancel} 
\usepackage{amsthm}
\usepackage{xifthen}
\usepackage{pdfpages}
\usepackage{calc}
\usepackage{svg}
\usepackage{titling}
\usepackage{array}

\newcommand{\incfig}[1]{%
\def\svgscale{1}
\import{./figures/}{#1.pdf_tex}
}
\newcommand\vertarrowbox[3][6ex]{%
  \begin{array}[t]{@{}c@{}} #2 \\
  \left\downarrow\vcenter{\hrule height 0.2pt}\right.\kern-\nulldelimiterspace\\
  \makebox[0pt]{\scriptsize#3}
  \end{array}%
  }

\begin{document}
\theoremstyle{plain}
\newtheorem{thm}{Theorem}[chapter]
\newtheorem{lem}[thm]{Lemma}
\newtheorem{prop}[thm]{Proposition}
\newtheorem*{cor}{Corollary}


\theoremstyle{definition}
\newtheorem{defn}[thm]{Definition}
\newtheorem{conj}{Conjecture}[section]
\newtheorem{exmp}[thm]{Example}

\theoremstyle{remark}
\newtheorem*{rem}{Remark}
\newtheorem*{note}{Note}


\title{Calculus I}
\author{SyG}
\maketitle


\chapter{Introduction}
\begin{defn}Mathematics (according to Oxford Eng. Dictionary) \end{defn}
The abstract science which investigates deductively the conclusions implicit in the 
elementary conception of spacial and numerical relations.
This science can be divided in 6 main topics:
\begin{enumerate}
  \item \textbf{Foundations:} logic, set theory, proof theorems, etc.
  \item \textbf{Algebra:} numbers, arithmetical operations, order theorems. 
  \item \textbf{Analysis:} differentiation, integration, measure, etc.
  \item \textbf{Geometry and topology:} proporties of space, shape, position ?? of figures.
  \item \textbf{Combinatories:} graph theory, partition theory, etc.
  \item \textbf{Applied Mathematics:} computational sciences, probability, the range of applications
  of Mathematics is wide, such as:
  \begin{enumerate}
    \item \textbf{Banking and Finance:} Black and Schole ??
    \item \textbf{Aeronautical engineering:} Fluid mechanics, shape design.
    \item \textbf{Chemistry:} Models for proteing folding, thermodynamics.
    \item \textbf{Informatic:} Cryptography, computational algebra, parallel programming, etc. 
  \end{enumerate}
\end{enumerate}


\subsection*{Summary of the program: 1st Semester}
\begin{itemize}
  \item Real numbers
  \item Sequence and series of numbers
  \item Continuity and limit
  \item Differentiation
  \item Integration
\end{itemize}

\chapter{Real numbers and some basic concepts}

\section{Set of points}
We recall here some basical concepts:


\end{document}