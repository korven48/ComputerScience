\documentclass[12pt, a4paper]{book}
\usepackage[top=2 cm, bottom=2 cm, left=2 cm, right=2 cm]{geometry}
\usepackage[T1]{fontenc}
\usepackage[utf8]{inputenc}
\usepackage{amsmath}
\usepackage{amssymb}
\usepackage{mathtools}
\usepackage{amsthm}
\usepackage[english]{babel}
\usepackage{parskip}
\usepackage{float}
\usepackage{physics}
\usepackage{graphicx}
\usepackage{multicol}
\usepackage{import}
\usepackage{cancel} 
\usepackage{amsthm}
\usepackage{xifthen}
\usepackage{pdfpages}
\usepackage{calc}
\usepackage{svg}
\usepackage{titling}
\usepackage{array}

\newcommand{\incfig}[1]{%
\def\svgscale{1}
\import{./Graphics/}{#1.pdf_tex}
}
\newcommand\vertarrowbox[3][6ex]{%
  \begin{array}[t]{@{}c@{}} #2 \\
  \left\downarrow\vcenter{\hrule height 0.2pt}\right.\kern-\nulldelimiterspace\\
  \makebox[0pt]{\scriptsize#3}
  \end{array}%
  }

\begin{document}
\theoremstyle{plain}
\newtheorem{thm}{Theorem}[chapter]
\newtheorem{lem}[thm]{Lemma}
\newtheorem{prop}[thm]{Proposition}
\newtheorem*{cor}{Corollary}


\theoremstyle{definition}
\newtheorem{defn}[thm]{Definition}
\newtheorem{conj}{Conjecture}[section]
\newtheorem{exmp}[thm]{Example}

\theoremstyle{remark}
\newtheorem*{rem}{Remark}
\newtheorem*{note}{Note}
\newtheorem*{notation}{Notation}


\title{Calculus I}
\author{SyG}
\maketitle


\chapter{Introduction}
\begin{defn} \textbf{Mathematics (according to Oxford Eng. Dictionary)} \end{defn}
The abstract science which investigates deductively the conclusions implicit in the 
elementary conception of spacial and numerical relations.
This science can be divided in 6 main topics:
\begin{enumerate}
  \item \textbf{Foundations:} logic, set theory, proof theorems, etc.
  \item \textbf{Algebra:} numbers, arithmetical operations, order theorems. 
  \item \textbf{Analysis:} differentiation, integration, measure, etc.
  \item \textbf{Geometry and topology:} proporties of space, shape, position of figures.
  \item \textbf{Combinatories:} graph theory, partition theory, etc.
  \item \textbf{Applied Mathematics:} computational sciences, probability, the range of applications
  of Mathematics is wide, such as:
  \begin{enumerate}
    \item \textbf{Banking and Finance:} Black-Scholes equation.
    \item \textbf{Aeronautical engineering:} Fluid mechanics, shape design.
    \item \textbf{Chemistry:} Models for protein folding, thermodynamics.
    \item \textbf{Informatic:} Cryptography, computational algebra, parallel programming, etc. 
  \end{enumerate}
\end{enumerate}


\subsection*{Summary of the program: 1st Semester}
\begin{itemize}
  \item Real numbers
  \item Sequence and series of numbers
  \item Continuity and limit
  \item Differentiation
  \item Integration
\end{itemize}

\chapter{Real numbers and some basic concepts}

\section{Set of points}
We recall here some basical concepts:
\begin{defn}
  A \textbf{set} is a \textbf{collection of distinct objects.}
\end{defn}
\begin{exmp}
  2, 5, 7 are different objects (numbers). They can compose the set $\{2, \ 5, \ 7 \}$, where $\{ \ldots \}$ denotes the
  set composed by the objects $\ldots$.
\end{exmp}

\begin{note}
  If an object $x$ is a member of a set $\theta$, we denote: 
  $$ x \in \theta, \ \text{else we denote } x \notin \theta$$

  \begin{exmp}
    \[
      \theta = \{0, \ 5, \ 7 \}, \ \ \text{if }x=5 \text{ and } y=9:
    \]
    \[
      x \in \theta \text{ and } y \notin \theta
    \]
  \end{exmp}
\end{note}

\begin{rem}
  A set cannot have two times the same object.
\end{rem}

\begin{defn}
  Considering two sets A and B. If every element of A is a member of B, A is said to be a \textbf{subset} of B, and we denote:
  \boldmath $$ A \subseteq B$$, else we denote 
  $$ A \nsubseteq B $$ 
  Furthermore, if it exists at least one element of B which is not a member of A (A is strictly in B), A is said to be a \textbf{proper subset} of B, and we denote

  $$ A \subset B $$. \unboldmath
  
\end{defn}

\begin{exmp}
  \[
    A= \{ 1, \ 2, \ 3\} 
  \]
  \[
    B= \{ 0, \ 1, \ 2, \ 3, \ 4 \}
  \]
  \[
    C= \{ 0, \ 1, \ 2 \}
  \]
  \[
    \therefore
    A \subseteq A, \ \ \ A \subset B, \ \ \ A \nsubseteq C
  \]
\end{exmp}

\begin{defn}
  \textbf{Set operators}
  Let A and B be two sets. 
\end{defn}
\boldmath
\subsubsection*{Union: $\cup$}

The \textbf{union} of $A$ and $B$ is the set

\[ 
  A \cup B = \{x/x \in A \vee x \in B \}
\]

\subsection*{Intersection: $\cap$}

The \textbf{intersection} of $A$ and $B$ is the set
\[
  A \cap B = \{x/x \in A \wedge x \in B \}
\]

\subsection*{Complement: $\backslash$}

\[
  A \; \backslash \; B = \{x/x \in A \wedge x \notin B \}
\]
\unboldmath
\begin{exmp}
  $A=\{3,5,7\}, B=\{5,7,10\}$
  \begin{itemize}
    \item $A \cup B = \{3,5,7,10\}$
    \item $A \cap B = \{5,7\}$
    \item $A \backslash B = \{3\}$
    \item $B \backslash A = \{10\}$
  \end{itemize}
\end{exmp}

\begin{rem}
  A set of one element is called a \textbf{singleton}.
\end{rem}

\subsubsection*{Geometrical representation}

\begin{figure}[H]
  \centering
  \incfig{Graphic1c}
\end{figure}

\subsubsection*{Properties (Morgan for sets): }

\begin{itemize}
  \item $A \backslash (B \cup C) = (A \backslash B) \cup (A \backslash C)$
  \item $A \backslash (B \cap C) = (A \backslash B) \cap (A \backslash C)$
\end{itemize}

Graphically (Venn diagrams):

\begin{figure}[H]
  \centering
  \incfig{Graphic2a}
\end{figure}

\begin{figure}[H]
  \centering
  \incfig{Graphic2b}
\end{figure}

\begin{note}
  In the case of various $n$ sets denoted by $A_1, \ A_2, \ A_n$, instead of writing:

  \[
    A_1 \cup A_2 \cup \ldots \cup A_n \text{ we write } \bigcup_{k=1}^{n} A_k
  \]
  or
  \[
    A_1 \cap A_2 \cap \ldots \cap A_n \text{ we write } \bigcap_{k=1}^{n} A_k
  \]
\end{note}

\begin{exmp}
  \[
    A_1=\{1, \ 2, \ 3 \}, \ \ \ A_2=\{ 5,\ 6, \ 7 \}, \ \ \ A_3= \{1, \ 5, \ 9 \}
  \]
  \[
    \bigcup_{k=1}^{3} A_k=A_1 \cup A_2 \cup A_3 = \{ 1, \ 2, \ 3, \ 5, \ 7, \ 9 \}
  \]
\end{exmp}

\begin{rem}
We can apply the same notation in case of infinite ($\infty$) numbers of a set $\{ A_1, \ldots, A_{100}, \ldots \}$.
\[
  \displaystyle\bigcup_{k=1}^{\infty}A_k \ \ \ \ \text{ and } \ \ \ \ \displaystyle\bigcap_{k=1}^{\infty}A_k
\]
\end{rem}
some examples and the concept of infinity will be defined in the next sections.

\begin{defn}
  The cartesian product of two sets A and B is denoted by \boldmath $A \times B$  and defined as:
  \[
    A \times B = \{ (a, \ b  ) \ | \ a \in A \text{ and } b \in B \}
  \]
  \unboldmath
  where (a, b) is called \textbf{ordened pair.} 
\end{defn}

\begin{exmp}
  \[
    A=\{1, \ 2, \ 3 \}, \ \ \ B=\{7, \ 9 \}
  \]
  \[
    A \times B= \{ (1,7), \ (1,9), \ (2,7), \ (2,9), \ (3,7), \ (3,9) \}
  \]
  The order is very important, it always goes first the elements of the first named set and then the ones of the second one.More properties of sets will be introduced later in this chapter.
\end{exmp}

\subsection*{Some common sets of real points}
Here we only introduce the set of points used in next chapters.

\begin{defn}
\end{defn}
\boldmath
  \begin{itemize}
    \item $\mathbb{R}=\{ \ldots, \ \ldots, \ -10, \ \ldots, \ -7, \ \ldots, \ 0, \ \ldots, \ 4, \ \ldots, 1000, \ \ldots  \}$ is called the set of \textbf{real numbers} which contains \textbf{all positive and negative numbers}.
    \item $\mathbb{N}= \{1, \ 2 , \ 3 , \ldots \}$ is called the set of \textbf{natural numbers} which contains \textbf{all the strictly positive integer numbers.}
    \begin{rem}
      $\mathbb{N^{*}}$ includes the 0.
    \end{rem}
    \item $\mathbb{Z}= \{\ldots, -5, \ -4, \ -3, \ -2, \ -1, \ 0, \ 1, \ 2, \ 3, \ldots  \}$ is called the set of \textbf{integer numbers} and contains the \textbf{positive and negative integers.}
    \item $\varnothing = \{\}$ the \textbf{empty set} represents the sets without any elements. 
    \begin{exmp}
      \[
        A=\{1, \ 4 \}, \ \ B=\{3, \ 4 \} \ \ \ \ A \cap B= \varnothing \text{, ie no coincidences between A and B}
      \]
    \end{exmp} 
    \item $\mathbb{Q}= \{x \in \mathbb{R} \ \vert \ x= \frac{m}{n}, \ m \in \mathbb{Z}, \ n \in \mathbb{Z} \text{ and } n \neq 0 \}$ is called the set of \textbf{rational numbers} and contains the \textbf{real numbers that can be written as a quotient of integer numbers}
    Numbers that don't belong in this set, ie $\sqrt{2}$ or $\pi$ are part of the \textbf{irrationals}, preferably noted as $\notin \mathbb{Q}$.
    \item \textbf{Odd=} $\{ x \in \mathbb{R} \ \vert \ \exists \ k \in \mathbb{N} \text{ st } x=2k+1 \}$ is the set of the \textbf{odd integer numbers}.
    \item \textbf{Even=} $\{ x \in \mathbb{R} \ \vert \exists \ k \ \in \mathbb{N} \text{ st } x=2k \}$ is the set of the \textbf{even integer numbers}.
    \item $\mathbb{C}= \{x +iy \ \vert \ x \in \mathbb{R} \text{ and } y \in \mathbb{R}  \}$ is the set of \textbf{complex numbers}. Note: i denotes the imaginary number that verifies $i^2=-1$.
  \end{itemize}
  \unboldmath

\begin{rem}
  \[
    \varnothing \subseteq \mathbb{N} \subseteq \mathbb{Z} \subseteq \mathbb{Q} \subseteq \mathbb{R} \subseteq \mathbb{C}
  \]
\end{rem}

\begin{defn}
  \boldmath
  A \textbf{relation}, noted by $\leq $, is a \textbf{total order} on a set S if it verifies:
  \begin{enumerate}
    \item \textbf{Reflexivity:} $\forall a \in S, \ a \leq a$
    \item \textbf{Antisymmetry:} $a \leq b$ and $ b \leq a \Rightarrow \ a=b$
    \item \textbf{Transitivity:} $a \leq b$ and $b \leq c \Rightarrow a \leq c$
    \item \textbf{Comparability:} $\forall \ a, b \in S$, either $a \leq b$ or $b \leq a$
  
    When Reflexivity, Antisymmetry and Transitivity occurs but no Comparability, then we have a \textbf{partial order}.
  \end{enumerate}
  \unboldmath
\end{defn}

\begin{exmp}
  \begin{itemize}
    \item The relation $\leq$ applied to $\mathbb{R}$ is a total order.
    \item The relation $\subset$ applied to a subset of $\mathbb{R}$ is \textbf{not} a total order. For example, $\{1, \ 2\}$ and $\{2, \ 4\}$ cannot be compared.  
  \end{itemize}
\end{exmp}

\begin{defn}
  A set plus a total order relation is called a \textbf{total ordered set}.
\end{defn}

\begin{exmp}
  \[
    (\mathbb{R}, \leq)
  \]

  Rules:
  \begin{enumerate}
    \item $a=b$
    \item $a<b$ or $a>b$ a strictly inferior (or superior) to b (not equal).
  \end{enumerate}
\end{exmp}

\begin{defn}
  \textbf{Infinity:} denoted by $\infty$, is an abstract concept ??? a limitless quantity (e.g. number).
\end{defn}

\subsubsection*{Properties:}

\begin{itemize}
  \item $\forall x \in \mathbb{R}, \ \ - \infty \leq x$ and $x \leq + \infty \ \therefore \ \mathbb{R}=(-\infty, +\infty).$
  \item $-\infty$ and $+\infty \notin \mathbb{R}$
\end{itemize}

\begin{defn}
  An \textbf{interval} is a real subset containing all the values between two given points, included or not. It can be of the type: 
  
  Let $a, \ b \in \mathbb{R}$:
  \begin{itemize}
    \item \textbf{Open interval}: $(a,b) = \{x\in \mathbb{R} / a < x < b\}$
    \item \textbf{Closed interval}: $[a,b] = \{x\in \mathbb{R} / a \leq x \leq b\}$
    \item \textbf{Left closed interval}: $[a,b) = \{x\in \mathbb{R} / a \leq x<b\}$
    \item \textbf{Left open interval}: $(a,b] = \{x\in \mathbb{R} / a < x \leq b\}$
  \end{itemize}
\end{defn}

\subsubsection*{Properties:} $\mathbb{R} = (-\infty , +\infty)$

***

\begin{defn}
  \textbf{Axiomatic definition of }$\mathbb{R}$.

  The real number system $(\mathbb{R}, +, \cdot, <)$ is a set where the following rules are defined.

  \begin{itemize}
    \item \textbf{Addition (+)}: a function
    \begin{align*}
      \mathbb{R}\times \mathbb{R} &\longrightarrow \mathbb{R} \\
      (x,y) &\longrightarrow x+y
    \end{align*}
    with the following properties:
    \begin{itemize}
      \item \textbf{Associativity}: $\forall x,y,z \in \mathbb{R}, \ (x+y)+z=x+(y+z)$
      \item \textbf{Commutativity}: $\forall x,y \in \mathbb{R}, \ x+y=y+x$
      \item \textbf{Identity element}: $\exists 0 \in \mathbb{R}/ \ 0+x=x+0=x$
      \item \textbf{Opposite element}: $\forall x \in \mathbb{R}, \exists! -x\in\mathbb{R}/ \ x+(-x)=(-x)+x=0$
    \end{itemize}

    \item \textbf{Multiplication ($\cdot$)}: a function 
    \begin{align*}
      \mathbb{R}\times \mathbb{R} &\longrightarrow \mathbb{R} \\
      (x,y) &\longrightarrow x \cdot y
    \end{align*}
    with the following properties:
    \begin{itemize}
      \item \textbf{Associativity}: $\forall x,y,z \in \mathbb{R}, \ (xy)z=x(yz)$
      \item \textbf{Commutativity}: $\forall x,y \in \mathbb{R}, \ xy=yx$
      \item \textbf{Identity element}: $\exists 1 \in \mathbb{R}/ \ 1\cdot x=x\cdot 1=x$
      \item \textbf{Inverse element}: $\forall x \in \mathbb{R}, \exists! \frac{1}{x}\in\mathbb{R}/ \ x\cdot \frac{1}{x} = \frac{1}{x} \cdot x = 1$
      \item \textbf{Distributivity}: $\forall x,y,z \in \mathbb{R}\backslash \{0\}, \ x(y+z)=xy + xz$
    \end{itemize}

    \item The field $(\mathbb{R},+,\cdot)$ is ordered:
    \begin{itemize}
      \item $\geq$ is a total order.
      \item $\forall x,y,z\in\mathbb{R}, \ x\geq y \Rightarrow x+y \geq y+z$
      \item $\forall x,y \geq 0, \ xy\geq 0$
    \end{itemize}

    \item The order is \textbf{Dedekind complete} (the supremum property):
    
    $A \neq \emptyset, \ A \subseteq \mathbb{R} \wedge \exists k\in\mathbb{R} / \forall a\in A, a\leq k$ (where $k$ is called \textit{upper bound}) $\Rightarrow \exists \alpha$ denoted $\sup A$ and called least upper bound, such that $\forall a\in\mathbb{A}, a\leq\alpha$ and $\forall k\in\mathbb{R}$ upper bound of $A,\ \alpha \leq k$.
  \end{itemize}
\end{defn}

\begin{rem}
  $\mathbb{N}$ cannot be defined axiomatically (e.g. $0\notin\mathbb{N}$)
\end{rem}


\section{Mathematical Functions}

\begin{defn}
  Let $A$ and $B$ being two sets. A function from $A$ to $B$, is a relation between $A$ and $B$, denoted by $f:A\rightarrow B$, such that $\forall a\in\mathbb{A}, \ \exists!\ b\in B / \ f(a) = b$.

  The elements of $A$ are called arguments of $f$. The element $b\in B$ such that $f(a) = b$, with $a \in A$ is called value at $a$ or image of $a$ under $f$.

  $A$ is called \textbf{domain} of $f$, $D(f)$, and $R(f) = \{b\in B/\ \exists a\in A /\ f(a)=b\}$ is the \textbf{range}.
\end{defn}

\begin{notation}
  \begin{align*}
    f:A &\longrightarrow B \\
    a &\longrightarrow f(a)
  \end{align*}
  We can write $f$ maps $A$ to $B$.
\end{notation}

\begin{exmp}
  \begin{align*}
    f:\mathbb{R} &\longrightarrow \mathbb{R} \\
    x &\longrightarrow x^2+1
  \end{align*}
  $f(1) = 1^2+1 = 2$

  $f(2) = 2^2+1 = 5$

  ***
\end{exmp}

\begin{defn}
  The graph of a function is its set of ordered pairs $F=\{(a,f(a)),\ \forall a\in A\}$
  \begin{align*}
    f:\mathbb{R} &\longrightarrow \mathbb{R} \\
    x &\longrightarrow x^2+1
  \end{align*}

  ***
\end{defn}

\begin{defn}
  \begin{itemize}
    \item If $E \subseteq A$, the image of $E$ under $f$ is $f(E) = \{f(x)/\ x\in E\}$.
    \item If $H \subseteq B$, the preimage of $H$ under $f$ is $f^{-1}(H) = \{x\in A/\ f(x)\in H\}$
  \end{itemize}
\end{defn}

\begin{exmp}
  $A=B=\mathbb{R},\ f(x) = x^2$.
  \begin{itemize}
    \item $E = [0,2] \subset \mathbb{R} \quad f(E) = [0,4]$
    \item $H = \{4,9\} \subset \mathbb{R} \quad f^{-1}(H) = \{-3,-2,2,3\}$
  \end{itemize}
\end{exmp}

\begin{defn}
  $f:A\longrightarrow B$
  \begin{itemize}
    \item $f$ is called \textbf{inyective} if $\forall a_1,a_2 \in A, f(a_1)=f(a_2) \Rightarrow a_1=a_2$
    \item $f$ is called \textbf{suryective} if $\forall b \in B,\ \exists a\in A /\ f(a)=b$
    \item $f$ is called \textbf{biyective} if $f$ is inyective and suryective.
  \end{itemize}
\end{defn}

\begin{exmp}
  \begin{itemize}
    \item 
    \begin{align*}
      f:\mathbb{R} &\longrightarrow \mathbb{R} \\
      x &\longrightarrow x^2
    \end{align*}
    It's not inyective, since $f(1) = f(-1)$, and it's not suryective since $-1\in\mathbb{R} \wedge \nexists a \in \mathbb{R}/\ f(a)=-1$
  \end{itemize}
\end{exmp}

\end{document}